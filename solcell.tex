\documentclass[a4paper]{article}
\usepackage[utf8]{inputenc}
\usepackage[intlimits]{amsmath}
\usepackage{amsthm}
\usepackage{amsfonts}
\usepackage{enumerate}
\begin{document}
\title{IM2601 Solid State Physics \\ Lab exercise 3: Photovoltaic effect: Diode IV-characteristics}
\author{Karl Amundsson \\ Alexander Bielik \\ Hannes Lindström}
\date{\today}
\maketitle
\newpage
\tableofcontents
\newpage
\section{Introduction}
This report concerns itself with the solar cell, which is a very large $p$-$n$ junction.
A $p$-$n$ junction is a semiconductor, comprised of two parts that are denoted by $p$ and $n$ respectively.
The $p$-side is doped with acceptor atoms and the other with with donor atoms
\begin{table}
  \centering
  \begin{tabular}{|c|c|}
    \hline
    Distance [cm] & $P_{max}$ [W] \\
    12.5 & 0.07983 \\
    17.5 & 0.05617 \\
    22.5 & 0.04975 \\
    27.5 & 0.04169 \\
    32.5 & 0.03914 \\
    37.5 & 0.03278 \\
    42.5 & 0.02902 \\
    47.5 & 0.02727 \\
    52.5 & 0.02393 \\
    57.5 & 0.02159 \\
    62.5 & 0.02100 \\
    \hline
  \end{tabular}
  \caption{Maximum power for different distances}
  \label{tab:<+label+>}
\end{table}<++>
\end{document}
